\chapter{Introduction}
\label{cap:introduction}

Since the development of the Internet as a global network, it has always hosted large amounts of documents and written information. As it grew, different ways had to be found to organize that information so that it could be used as a service to any user. The Semantic Web, whose characteristics we will explain later, is one of the methods that the entire Internet was able to adapt when it came to giving a meaning and a utility to that information.\\

The Semantic Web tries to link all that information in order to give it context and accessibility. To do this, it adopts the method of Linked Data, which, referring to its own name, connects or links different data establishing a relationship. The main objective of this is to give computers the ability to navigate and access most of the data stored on the Internet so that they are useful and not lost in a sea of information.\\

It is through this concept that search engines can be helped by this technology, as we provide machines the ability to navigate over those relationships and understand data that we require or could be useful to us.\\

In this project we will try to use Linked Data technology to enrich a music recommender system, providing explanations that justify its recommendations.\\

Usually a recommendation in certain softwares is not associated to a detailed explanation of it. For this reason we have tried to develop a system based on different explanations that give the user a global idea of why one song has been recommended based on another. The purpose is to represent that model visually because we think it is the most effective way to convey the idea to a person.\\

\section{Motivation}

The goal of this project is to give explanations to a music recommender. We start from a music recommender that provides a song to the user based on another song he has listened to, allegedly because there is some kind of relationship between them. Using that premise, we want to explain \textbf{why} those two songs are related or, in other words, how they resemble each other. To achieve this goal we will use the semantic web.\\

As we mentioned, the Semantic Web is a way to improve the connection between the data stored in the net, making the Internet more useful and accessible to people. The problem is that it is a practice that is not widespread enough to represent a real advance in our way of relating to the network yet, because it requires making changes in the way information is stored. In this paper we will try to show the potential of the Semantic Web.\\

The main point is to be able to collect all the connections that appear between all the information that we can get from each of the two musical themes or songs. For this, we must study that entity and all related: artist who performs the song, its musical genre, the members of its band, its record label, etc. and in turn the relations between them, until we have enough connections that can guarantee that those two songs have some relationship.\\

The connections obtained are explanations, which justify the relationship established by the system between two entities.\\

\section{Objectives}

The main objectives of this project are:

\begin{itemize}
\item To make a study on the Semantic Web and the data models that allow it, with the aim of developing later an application that exemplifies the practical applications that this Semantic Web can have.\\

\item To determine which properties may be most useful to establish explanations for the relationship between two provided songs, even if they are not very similar musically but share other interesting elements. This is the premise of our project and, thanks to our research, we will be able to determine which properties are most important when obtaining meaningful explanations in the domain.\\

\item To design a system that combines all the explanations with sense and logic, that can be hierarchical when it comes to organizing each of the explanations that we established and that is easy to represent visually.\\

\item To study different technologies that help us in the elaboration of the aforementioned application, in addition to expanding our knowledge in areas of computer science that we have not necessarily explored during the degree.\\

\item To design a visual interface of explanations, using iterative development that allows us to obtain a satisfactory result. This development includes the implementation of the design in our application.\\

\item Once completed, draw conclusions about the work done and set tasks to be performed in a future development.\\
\end{itemize}

\section{Work plan}

After explaining the motivation of this project and the objectives proposed for it, we proceed to detail the stages crossed in the development process.\\

Initially we became familiar with the Semantic Web and Linked Data, as well as some of the related technologies. This research has been detailed in Chapter~\ref{cap:estadoDeLaCuestion}.\\

Then we established the list of explanations that we would use in the final version of the application, which is collected and explained in Chapter~\ref{cap:explicaciones}.\\

Later we started the development of the application, starting with the design of the explanation interface (documented in Chapter~\ref{cap:interfaz}) and continuing with the implementation, including its architecture and functionality. This process is explained in the Chapter~\ref{cap:implementacion}.\\

Finally, we reflected on the work done and the subject studied throughout the project. These conclusions are noted in Chapter~\ref{cap:conclusions}, as well as the future work we should do if we continued to develop the application.\\