\chapter{Introducción}
\label{cap:introduccion}

Desde el desarrollo de Internet como una red global, siempre ha albergado grandes cantidades de documentos e información escrita. A medida que fue creciendo, hubo que encontrar distintas maneras de organizar esa información para que pudiera ser utilizada como un servicio a cualquier usuario. La Web Semántica, cuyas características explicaremos más tarde, es uno de los métodos que toda la red global de Internet pudo adaptar a la hora de darle un sentido y una utilidad a esa información.\\

Básicamente, la Web Semántica trata de enlazar toda esa información con tal de darle contexto y accesibilidad. Para ello adopta el método de Linked Data, el cual haciendo referencia a su propio nombre, conecta o une distintos datos estableciendo una relación. El principal objetivo de todo esto es que los ordenadores puedan navegar y acceder a la mayor parte de los datos almacenados en Internet para que estos sean útiles y no queden perdidos en un mar de información.\\

Es mediante este concepto como los motores de búsqueda pueden ser ayudados por esta tecnología, ya que dotamos a las máquinas de la capacidad para navegar sobre esas relaciones y entender datos que nosotros requerimos o que podrían sernos útiles.\\

En este trabajo trataremos de emplear la tecnología de Linked Data para enriquecer un sistema de recomendación de música. Normalmente una recomendación en determinados softwares no viene ligada a una explicación más o menos detallada de la misma. Por este motivo hemos intentado desarrollar un sistema basado en distintas explicaciones que proporcionen al usuario una idea global del porqué una cación ha sido recomentada en base a otra. La finalidad es representar ese modelo gráficamenete por es lo más efectivo a la hora de plasmar una idea general sobre cualquier concepto en una persona.\\

\section{Motivación}

La razón de ser de este proyecto consiste en dar explicaciones a un recomendador de música. Partimos de un recomendador de música que proporciona una canción al usuario en base a otra canción que haya escuchado, supuestamente porque tienen algún tipo de relación entre ellas. Utilizando esa premisa como base, queremos explicar \textbf{por qué} esas dos canciones están relacionadas o, dicho de otra forma, en qué se parecen. Para alcanzar esa meta haremos uso de la web semántica.\\

Como ya hemos comentado, la Web Semántica es una manera de mejorar la conexión entre los datos guardados en la red, haciendo que Internet sea más útil y accesible para las personas. El problema es que es una práctica que aún no está lo bastante extendida para que suponga un verdadero avance en nuestra forma de relacionarnos con la red, pues requiere hacer cambios en la forma en que se almacena la información. En este trabajo intentaremos mostrar el potencial de la Web Semántica.\\

La idea principal es poder recoger todas las conexiones que aparezcan entre toda la información que podamos recoger sobre cada uno de los dos temas musicales o canciones. Para ello, deberemos estudiar ese ente y todos los relacionados: artista que interpreta la canción, su género musical, los integrantes de su grupo, su sello discográfico, etc. y a su vez las relaciones entre estos, hasta disponer de suficientes conexiones que nos puedan asegurar que esas dos canciones tienen o no algo que ver. Las conexiones obtenidas serán explicaciones, que justifican la relación entre dos entidades establecida por el sistema.\\

Es una idea muy interesante, el poder esquematizar valores y explicaciones que han influido a la hora de recomendar dos canciones que a priori, son similares entre sí o incluso que no lo son tanto, pero pueden tener puntos de unión poco evidentes. El poder relacionar ambos casos y representarlo gráficamente, de forma que el usuario pueda entenderlo y hacerse una idea de cómo las explicacones establecen puntos de unión, supone un verdadero reto. \\

\section{Objetivos}

Los objetivos principales de este trabajo son los siguientes:\\

\begin{itemize}
\item Hacer un estudio sobre la Web Semántica y los modelos de datos que la permiten, con el objetivo de desarrollar más tarde una aplicación que ejemplifique las aplicaciones prácticas que puede tener esta Web Semántica.\\

\item Determinar qué propiedades pueden llegar a ser más útiles para establecer explicaciones para la relación entre dos canciones proporcionadas, incluso si no son muy parecidas musicalmente pero comparten otros elementos interesantes. Esta es la premisa de nuestro trabajo y, gracias a nuestro estudio, podremos determinar qué propiedades son más importantes a la hora de obtener explicaciones significativas en el dominio.\\

\item Diseñar un sistema que una todas las explicaciones con sentido y lógica, que pueda ser jerárquico a la hora de guardar cada una de las explicaciones que establecemos y que sea sencillo de representar gráficamente.\\

\item Estudiar diferentes tecnologías que nos ayuden en la elaboración de la aplicación mencionada, además de ampliar nuestros conocimientos en áreas de la informática que no hemos explorado necesariamente durante el grado.\\

\item Diseñar una interfaz visual de explicaciones, utilizando un desarrollo iterativo que nos permita obtener un resultado satisfactorio. Este desarrollo incluye la implementación del diseño en nuestra aplicación.\\

\item Una vez terminado, extraer conclusiones sobre el trabajo realizado y establecer tareas a realizar en un desarrollo futuro.\\
\end{itemize}


\section{Plan de trabajo}

Tras haber explicado la motivación de este trabajo y los objetivos propuestos para el mismo, pasamos a detallar las etapas atravesadas en el proceso de desarrollo.\\

Inicialmente nos hemos familiarizado con la web semántica y Linked Data, así como con algunas de las tecnologías relacionadas. Este estudio se ha detallado en el Capítulo~\ref{cap:estadoDeLaCuestion}.\\

A continuación establecimos la lista de explicaciones que utilizaríamos en la versión final de la aplicación, la cual viene recogida y explicada en el Capítulo~\ref{cap:explicaciones}.\\

Posteriormente iniciamos el desarrollo de la aplicación, empezando con el diseño de la interfaz de explicaciones (documentado en el Capítulo~\ref{cap:interfaz}) y continuando con la implementación, incluyendo su arquitectura y funcionalidad. Este proceso está explicado en el Capítulo~\ref{cap:implementacion}.\\

Por último, dedicamos un tiempo a reflexionar sobre el trabajo realizado y el ámbito estudiado a lo largo del proyecto. Estas conclusiones están anotadas en el Capítulo~\ref{cap:conclusiones}, al igual que el trabajo futuro que deberíamos realizar si continuásemos con el desarrollo de la aplicación.\\
