\chapter{Introducción}
\label{cap:introduccion}

Desde el desarrollo de Internet como una red global, siempre ha albergado grandes cantidades de documentos e información escrita. A medida que fue creciendo, hubo que encontrar distintas maneras de organizar esa información para que pudiera ser utilizada como un servicio a cualquier usuario. La Web Semántica, cuyas características explicaremos más tarde, es uno de los métodos que toda la red global de Internet pudo adaptar a la hora de darle un sentido y una utilidad a esa información.\\

Básicamente, la Web Semántica trata de enlazar toda esa información con tal de darle contexto y accesibilidad. Para ello adopta el método de Linked Data, el cual haciendo referencia a su propio nombre, conecta o une distintos datos estableciendo una relación. El principal objetivo de todo esto es que los ordenadores puedan navegar y acceder a la mayor parte de los datos almacenados en Internet para que estos sean útiles y no queden perdidos en un mar de información.\\

Hay varias maneras de poder organizar y conectar todo ese contenido conocidas como sistemas ontológicos. La ontología es una rama filosófica que estudia los entes y sus conexiones y aquí es aplicada en su punto más técnico. El método ontológico que nosotros usaremos es el modelo RDF, uno de los más usados y extendidos.\\

Es mediante este concepto como los motores de búsqueda pueden ser ayudados por esta tecnología, ya que dotamos a las máquinas de la capacidad para navegar sobre esas relaciones y entender datos que nosotros requerimos o que podrían sernos útiles.\\

\section{Motivación}

La razón de ser de este proyecto consiste en dar explicaciones a un recomendador de música. Partimos de un recomendador de música que proporciona dos canciones, las cuales tienen algún tipo de relación entre ellas. Utilizando esa premisa como base, queremos explicar \textbf{por qué} esas dos canciones están relacionadas o, dicho de otra forma, en qué se parecen. Para alcanzar esa meta haremos uso de la web semántica.\\

Como ya hemos comentado, la Web Semántica es una manera de mejorar la conexión entre los datos guardados en la red, haciendo que Internet sea más útil y accesible para las personas. Se trata de un método que permite que navegar por la red sea más intuitivo y reduce la cantidad de información que acaba volviéndose inaccesible para aquellos usuarios para los que resulta relevante.\\

El problema es que es una práctica que aún no está lo bastante extendida para que suponga un verdadero avance en nuestra forma de relacionarnos con la red, pues requiere hacer cambios en la forma en que se almacena la información. En este trabajo intentaremos mostrar el potencial de la Web Semántica.\\

La idea principal es poder recoger todas las conexiones que aparezcan entre toda la información que podamos recoger sobre cada uno de los dos temas musicales o canciones. Para ello, deberemos estudiar ese ente y todos los relacionados: artista que interpreta la canción, su género musical, los integrantes de su grupo, su sello discográfico, etc. y a su vez las relaciones entre estos, hasta disponer de suficientes conexiones que nos puedan asegurar que esas dos canciones tienen o no algo que ver.\\

\section{Objetivos}

Los objetivos principales de este trabajo son los siguientes:\\

\begin{itemize}
\item Demostrar la potencia y las conexiones que se pueden llegar a crear entre dos elementos usando el formato RDF en el contexto de la Web Semántica. La Web Semántica es un concepto que abarca muchísima información con conexiones constantes que no para de crecer. Queremos adaptarnos a ese sistema y aplicarlo a un dominio con el que estuviésemos cómodos y despertara nuestro interés.\\

\item Establecer una relación entre dos canciones que, a priori, una persona no podría obtener sin hacer un estudio amplio de ello, empleando para ello explicaciones propias que compartan esas dos canciones. Y, sobre todo, intentar relacionar canciones que no sean tan parecidas musicalmente pero compartan otros elementos que puedan causar más curiosidad.\\

\item Crear una aplicación web que pueda tener un uso habitual en la vida cotidiana y que aporte nuevos conocimientos a los posibles usuarios que la utilicen.\\

\item Utilizar distintas tecnologías de forma que tengamos un resultado final operativo y simple para que cualquier usuario, tenga o no experiencia en el manejo de aplicaciones, sea capaz de usarlo.\\

\item Trabajar con un dominio durante todo el proyecto que nos apasiona tanto a nosotros como a la mayoría de las personas como es la música, ya que al fin y al cabo la aplicación final recopila datos sobre canciones que un usuario no podría saber sin hacer un estudio previo.\\
\end{itemize}

\section{Plan de trabajo}

El punto inicial del proyecto fue estudiar todas las nueva tecnologías nuevas y herramientas que íbamos a necesitar, al menos, para el proceso de obtención y estudio los datos.\\

Primeramente hicimos una pequeña limpieza del dataset, ya que había valores sucios y nulos. Hicimos unas pequeñas agrupaciones, por popularidad y por género. La popularidad porque las canciones más populares serían las que estarían más documentadas en Wikidata y el género debido a que queríamos saber qué tipo de canciones predominaban en nuestra muestra.\\

Más tarde comenzamos a ejecutar pequeñas consultas con el sistema RDF en la herramienta propia de Wikidata. Aquí aprendimos en que se diferenciaban las consultas que más tarde usaríamos en nuestra la librería del propio SPARQL. En este punto ya pudimos observar la cantidad de información que podíamos obtener con una simple consulta y a la vez los datos tan específicos que también nos daba la oportunidad de conseguir, dándonos cuenta de la potencia de este lenguaje.\\

Una vez obtuvimos cierta soltura con Wikidata y las nuevas tecnologías que íbamos a usar, pasamos a buscar una librería lo suficientemente completa y potente que pudiese servirnos para ejecutar las consultas deseadas y en un tiempo aceptable. Después de estudiar algunas opciones, finalmente escogimos SPARQLWrapper debido a que nos permitía la ejecución de todo tipo de consultas y además nos facilitaba la conexión al punto de obtención.\\

Continuando con el proceso comenzamos el algoritmos y el conjunto de clases y librerías que nos permitirían obtener los datos y establecer las conexiones necesarias para dotar de una posible relación a dos canciones. Tuvimos varias versiones, ya que siempre incluíamos nuevas partes del código, nuevas propiedades a estudiar, distintas formas de devolver el resultado etc… Al final de este punto ya éramos capaces de obtener un gran conjunto de relaciones en distintos niveles del estudio y un formato que las organizaba y las clasificaba.\\