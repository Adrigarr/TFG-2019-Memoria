\chapter{Explicaciones}
\label{cap:descripcionTrabajo}

El objetivo de nuestro trabajo consiste en explicar las relaciones entre dos canciones dadas por un recomendador de música que trabaja con un dataset concreto. De esta manera, llamaremos \textbf{explicación} a cada factor común que compartan ambas canciones, siendo principalmente propiedades que poseen las propias canciones o los valores de otras propiedades previas. A lo largo de este capítulo concretaremos todas las explicaciones que vamos a estudiar.

En una primera fase, generaremos explicaciones básicas para relacionar diferentes tipos de canciones, por ejemplo: género, artista, álbum, etc. Asignaremos una complejidad de k=1 a las explicaciones directas. Estas explicaciones son una relación directa entre dos canciones relacionadas por una propiedad.\\

Después generaremos explicaciones más complejas que normalmente un humano pasaría por alto. Estas explicaciones complejas pueden estar formadas por relaciones indirectas entre los datos de una forma que se puede representar con un grafo. Estas explicaciones pueden tener un nivel de complejidad diferente (k= 2, 3, 4...) dependiendo de cuántas aristas del grafo separen ambos elementos.\\


\section{Explicaciones directas}

En esta sección vamos a recopilar todas la propiedades propias de la canción en sí. Esto significa que son propiedades directamente relacionadas con la canción siguiendo el modelo RDF que podemos obtener con una consulta básica. Aquí encontraremos propiedades como el artista que interpreta la canción, la compañía discográfica que la ha publicado o los premios que ha recibido, entre otras.


\subsection*{Artista}

El artista es una de las explicaciones más obvias pero también es una de las más importantes. Esta explicación hace referencia a cuando dos temas son interpretados por el mismo artista, por lo que tienen una clara relación directa ya que suele existir similitud entre las canciones de un artista. Además, a partir del artista podemos obtener otras relaciones más complejas en función de los datos de este mismo.

Como ejemplo podemos tomar los temas \textit{One More Time} y \textit{Get Lucky}. Ambos son interpretados por el dúo musical \textbf{Daft Punk}, así que podemos explicar su relación gracias a este dato.

Para esta explicación utilizamos la propiedad ``performer'' (intérprete) de Wikidata. Siguiendo el modelo RDF, el sujeto es la canción (\textit{One More Time}, por ejemplo), el predicado es la propiedad intérprete y el objeto es el artista (\textbf{Daft Punk}). \\

\begin{figure}[h!]
	\centering
	\includegraphics[width = 0.9\textwidth]{Imagenes/Bitmap/Artista ejemplo.png}
	\caption{Ejemplo de explicación por artista}
	\label{fig:sampleImage}
\end{figure}

\subsection*{Álbum}

Una explicación muy potente será que ambas canciones pertenezcan al mismo álbum. A menudo esta explicación aparecerá acompañada de la explicación del artista y, en cualquier caso, la relación que existe entre dos temas del mismo álbum suele ser más estrecha debido a que poseen más puntos en común, como puede ser el género, la fecha o la temática.

Tomemos como ejemplo \textit{Smells Like Teen Spirit} y \textit{Come as You Are}. Estas dos canciones son muy cercanas porque tienen varios puntos en común, pero una de las explicaciones que podemos ofrecer es que ambas pertenecen al álbum \textit{Nevermind}, de \textbf{Nirvana}. \textit{About a Girl} es otro tema que se podría recomendar a raíz de \textit{Smells Like Teen Spirit}, pero es una peor elección que \textit{Come as You Are} porque pertenece a un álbum diferente.

Volvemos a utilizar Wikidata para obtener esta explicación, concretamente la propiedad ``part of'' (parte de). Esta propiedad se utiliza para varias cosas, entre ellas indicar los álbumes a los que pertenecen las canciones. Así obtenemos que \textit{Smells Like Teen Spirit} es ``parte de'' \textit{Nevermind}.\\

\begin{figure}[h!]
	\centering
	\includegraphics[width = 0.9\textwidth]{Imagenes/Bitmap/Álbum ejemplo.png}
	\caption{Ejemplo de explicación por álbum}
	\label{fig:sampleImage}
\end{figure}

\subsection*{Década}

Creemos que hay una mayor probabilidad de que exista una relación entre dos canciones que pertenezcan a la misma década. Esto se debe a que a lo largo del tiempo ha habido periodos marcados por uno o varios géneros musicales. Esto también ayuda a estudiar cómo estos distintos géneros están relacionados entre sí, lo cual es otro punto importante a tener en cuenta ya que hay géneros íntimamente relacionados entre sí: techno y house, heavy metal y thrash metal, etc.

Gracias a esta explicación, podemos relacionar dos canciones que a priori son muy distintas, como es el caso de \textit{September} de \textbf{Earth, Wind \& Fire} y \textit{Bohemian Rhapsody} de \textbf{Queen}. Estos temas no comparten algo tan básico como el artista o el género, pero ambos fueron publicados en los años 70 y por ello pueden resultar interesantes para un usuario que busque escuchar los ritmos de esa época.

Para obtener esta explicación, emplearemos la propiedad ``publication date'' (fecha de publicación) de las canciones en Wikidata y comprobaremos si las dos fechas se sitúan dentro de la misma década.\\

\begin{figure}[h!]
	\centering
	\includegraphics[width = 0.9\textwidth]{Imagenes/Bitmap/Década ejemplo.png}
	\caption{Ejemplo de explicación por década}
	\label{fig:sampleImage}
\end{figure}

\subsection*{Premios}

La siguiente explicación son los premios recibidos. Existe una variedad de premios compartidos por diferentes canciones. Algunos de estos premios son más específicos que otros, pero en cualquier caso pueden resultar una relación útil para nuestras explicaciones ya que son un reflejo de la repercusión de las canciones.

Así explicaríamos la relación entre \textit{Single Ladies (Put a Ring on It)} de \textbf{Beyoncé} y \textit{Rolling in the Deep} de \textbf{Adele}, ya que ambos temas han recibido el Premio Grammy a la canción del año.

Usaremos la propiedad ``award received'' (premio recibido) encontrada en Wikidata. De esta forma, \textit{Rolling in the Deep} sería el sujeto, ``award received'' sería el predicado y \textit{Grammy Award for Song of the Year} sería el objeto.\\

\begin{figure}[h!]
	\centering
	\includegraphics[width = 0.9\textwidth]{Imagenes/Bitmap/Premio ejemplo.png}
	\caption{Ejemplo de explicación por premio}
	\label{fig:sampleImage}
\end{figure}

\subsection*{Género}

La explicación Género es una de las más representativas ya que las personas se guían por el género que prefieren para escuchar canciones similares, aunque en ocasiones no deba ser así. Un usuario que sea fan del jazz querrá escuchar canciones de ese mismo género o géneros similares porque todas los temas de un mismo género comparten una serie de puntos en común como la elección de los instrumentos, temáticas similares en sus letras o patrones de composición.

Gracias al género, podemos explicar la relación entre canciones como \textit{Master of Puppets} de \textbf{Metallica} e \textit{Indians} de \textbf{Anthrax}, que pertenecen al \textbf{thrash metal}.

Para esta relación utilizamos la propiedad ``genre'' (género) de Wikidata. Así tenemos que la canción \textit{Master of Puppets} pertenece al género \textbf{thrash metal}.\\

\begin{figure}[h!]
	\centering
	\includegraphics[width = 0.9\textwidth]{Imagenes/Bitmap/Género ejemplo.png}
	\caption{Ejemplo de explicación por género}
	\label{fig:sampleImage}
\end{figure}

\subsection*{Compañía discográfica}

Compañía discográfica hace referencia a la compañía por la que ha firmado el artista para publicar un tema concreto. Hemos podido observar que una misma compañía puede firmar a artistas que a veces no tienen relación ninguna en cuanto al estilo musical, pero hay otras causas que sí los pueden relacionar indirectamente como la tendencia del momento, el target del público que generan, etc.

Para ilustrarlo con un ejemplo, tanto la canción \textit{Born in the U.S.A.} de \textbf{Bruce Springsteen} como la canción \textit{Ashes} de \textbf{Céline Dion} fueron publicadas por el sello discográfico \textbf{Columbia Records}.

Podemos alcanzar esta explicación gracias a la propiedad ``record label'' (sello discográfico) de Wikidata. \textit{Born in the U.S.A.} tiene un ``record label'' cuyo nombre (o valor) es \textbf{Columbia Records}.\\

\begin{figure}[h!]
	\centering
	\includegraphics[width = 0.9\textwidth]{Imagenes/Bitmap/Discográfica ejemplo.png}
	\caption{Ejemplo de explicación por compañía discográfica}
	\label{fig:sampleImage}
\end{figure}

\subsection*{Singles}

Existe también una cierta relación entre aquellos temas que sean singles o sencillos, así que consideramos esto como una explicación más. El razonamiento para esta decisión es que los singles son canciones que se publican de forma independiente por razones promocionales, por lo que suelen convertirse en los temas más populares y representativos del trabajo del artista. Por ello, pueden poseer más valor para el recomendador que otras canciones porque los singles tienen más probabilidades de coincidir con los gustos del usuario.

El tema \textit{Bad Guy} de \textbf{Billie Eilish} es un single, así que podemos decir que tiene cierta relación con \textit{Shallow} de \textbf{Lady Gaga} y \textbf{Bradley Cooper}, que también es un single.

Para esta explicación emplearemos la propiedad ``instance of'' (instancia de) en Wikidata y su valor ``single''. De esta forma comprobaremos que tanto \textit{Bad Guy} como \textit{Shallow} son ``instancia de'' ``single''.\\

\begin{figure}[h!]
	\centering
	\includegraphics[width = 0.9\textwidth]{Imagenes/Bitmap/Single ejemplo.png}
	\caption{Ejemplo de explicación por single}
	\label{fig:sampleImage}
\end{figure}

\subsection*{Idioma}

La explicación Idioma denota, como su nombre sugiere, que ambas canciones están en el mismo idioma o lenguaje. Existen numerosos ejemplos de temas musicales que están escritos en un idioma que no se corresponde con la lengua materna del artista que los publican, o sencillamente están escritas en un idioma que se habla en varias partes del mundo distintas, como el inglés o el español. En ciertos casos el compartir idioma es un indicativo de otras relaciones adicionales, como puede ser el caso de canciones que pertenecen a un folclore regional determinado. En cualquier caso el compartir lenguaje ya es una relación directa en sí misma.

Como ejemplo podemos tomar el tema \textit{Like a Virgin} de \textbf{Madonna} y \textit{Rocket Man} de \textbf{Elton John}, los cuales están escritos en inglés.

Para esta explicación emplearemos la propiedad ``language of work or name'' (idioma de la obra o del nombre) de Wikidata. Así tenemos que la canción \textit{Like a Virgin} tiene idioma ``inglés''.\\

\begin{figure}[h!]
	\centering
	\includegraphics[width = 0.9\textwidth]{Imagenes/Bitmap/Idioma ejemplo.png}
	\caption{Ejemplo de explicación por idioma}
	\label{fig:sampleImage}
\end{figure}


\section{Explicaciones indirectas}

La explicaciones indirectas nos van a ofrecer relaciones más complejas que nos proporcionen información que a simple vista no podríamos relacionar, ya que están relacionadas con sujetos externos como son el artista que interpreta la canción y el género de la canción.

El punto de partida principal serán las explicaciones directas. De cada una de las explicaciones directas podemos obtener más propiedades que indirectamente estarán relacionadas con la canción. Así pues, desde una canción podemos obtener su artista, a partir del artista los géneros musicales con los que está relacionado y pasamos a estudiar subgéneros de esos géneros obteniendo así un grafo en forma de árbol que puede dividirse en varios niveles. Podemos profundizar en estos niveles tanto como nosotros creamos conveniente. \\

A continuación enumeramos las explicaciones indirectas adicionales que buscamos para establecer la relación entre dos canciones. Cabe destacar que las dos principales ramas o propiedades directas que vamos a explorar son el género y el artista/grupo de la canción.

\subsection{Explicaciones basadas en el artista}

La primera propiedad de las canciones que hemos estudiado en profundidad es el artista. Se trata de uno de los aspectos que más influye en los temas musicales y, lo que es más importante, nos permite acceder a mucha información útil que no se relaciona de forma directa con la propia canción.

Las siguientes explicaciones se basan en propiedades de los artistas que interpretan las canciones analizadas:

\subsubsection*{País de origen (Artista)}

El primer estudio de relaciones indirectas viene dado por investigar y obtener propiedades del artista que interpreta la canción.

Para empezar podemos fijarnos en el país de origen del artista. Esta explicación puede ser especialmente útil para países relativamente pequeños donde la música nacional presenta unos patrones claros. Además, a lo largo de la historia en un país se genera una tendencia o nuevo género musical el cual es propio de ese país.

La canción \textit{Fire} de la banda \textbf{BTS} y \textit{Kill This Love} de \textbf{Blackpink} guardan cierta similitud porque ambos grupos son originarios de \textbf{Corea del Sur}.

Para comprobarlo, obtendremos la propiedad ``country of origin'' (país de origen) del artista que previamente ya hemos almacenado al estudiar las relaciones directas en el caso de las bandas o la propiedad ``country of citizenship'' (país de nacionalidad) en el caso de los artistas en solitario. \textit{Fire} tiene el ``performer'' \textbf{BTS}, cuyo ``country of origin'' es Corea del Sur.

\begin{figure}[h!]
	\centering
	\includegraphics[width = 1\textwidth]{Imagenes/Bitmap/País ejemplo.png}
	\caption{Ejemplo de explicación por país de origen}
	\label{fig:sampleImage}
\end{figure}

Cabe destacar que también hemos incluido una propiedad ``location of formation'' (lugar de formación) que hace referencia a la localidad de formación del grupo o artista, ejerciendo una selección más precisa pues nos da un núcleo de población más pequeño. Un ejemplo de ello podría ser dos míticas bandas de la ciudad de Boston: Aerosmith y Boston. Creemos que la unión de la ciudad natal con la música puede hacer a un usuario el escuchar grupos de su misma ciudad o estado.


\subsubsection*{Influencia}

La \textbf{influencia de los artistas} es otra explicación importante. A menudo el trabajo de un artista se ve influenciado por otros artistas de formas que no siempre están ligadas a un género musical concreto, así que podemos encontrar una relación entre dos canciones examinando estas influencias.

Tomando como ejemplo las canciones \textit{Billie Jean} de \textbf{Michael Jackson} y \textit{Paint It Black} de \textbf{The Rolling Stones}, podemos establecer una relación entre ellas al ver que ambos artistas fueron influenciados por la banda \textbf{The Beatles}.

Obtenemos esta explicación gracias a la propiedad ``influenced by'' (influenciado por) de Wikidata, además de la propiedad ``performer'' para relacionar el tema con su artista. Así averiguamos que \textit{Billie Jean} tiene de ``intérprete'' a \textbf{Michael Jackson}, quien fue ``influenciado por'' \textbf{The Beatles}.

\begin{figure}[h!]
	\centering
	\includegraphics[width = 1\textwidth]{Imagenes/Bitmap/Influencia ejemplo.png}
	\caption{Ejemplo de explicación por influencia}
	\label{fig:sampleImage}
\end{figure}

\subsubsection*{Integrantes}

Los \textbf{integrantes} son cada uno de los miembros que conforman un grupo. En caso de ser un único artista, esta propiedad es irrelevante ya que haríamos referencia a la propiedad directa de artista. Esta propiedad resulta útil en casos en los que miembros de la banda han cambiado de grupo a lo largo de su carrera musical. Esto nos da acceso a otros grupos que por valores o gustos musicales del artista nos hacen pensar que pueden ser muy parecidos.

Si tomamos los temas \textit{Lithium} de \textbf{Nirvana} y \textit{The Pretender} de \textbf{Foo Fighters} podemos observar que existe una relación entre ellos porque el músico \textbf{Dave Grohl} ha formado parte de ambas bandas.

Para comprobarlo usaremos la propiedad de Wikidata ``performer'' y después comprobaremos la propiedad ``has part'' (compuesto de) para ver todos los miembros de la banda y buscar coincidencias. De este modo, podemos ver que \textit{The Pretender} tiene un ``performer'' que es \textbf{Foo Fighters}, y a su vez \textbf{Foo Fighters} ``has part'' \textbf{Dave Grohl}.\\\\\\

\begin{figure}[h!]
	\centering
	\includegraphics[width = 0.8\textwidth]{Imagenes/Bitmap/Integrante ejemplo.png}
	\caption{Ejemplo de explicación por integrantes}
	\label{fig:sampleImage}
\end{figure}

\subsubsection*{Tipo de voz}

Siguiendo con el estudio de los artistas, el \textbf{tipo de voz} de los vocalistas es una buena explicación para relacionar dos canciones. El tipo de voz influye en el sonido general del tema y puede ser especialmente determinante en ciertos géneros musicales. Por limitaciones técnicas, solo aplicaremos esta explicación con artistas en solitario.

La voz de un artista puede encajar en más de un tipo, como veremos en el siguiente ejemplo, aunque en esta explicación buscamos que coincidan en al menos un tipo. La cantante \textbf{Lady Gaga} se asemeja a \textbf{Amy Winehouse} por su tipo de voz, ya que ambas son \textit{mezzo-soprano} y \textit{contralto}. Así podemos explicar la relación entre dos canciones de estas artistas como \textit{Poker Face} y \textit{Rehab}.

Para esta explicación resulta útil la propiedad ``voice type'' (tipo de voz) en Wikidata. Partiendo de \textit{Poker Face}, utilizamos la propiedad ``performer'' para llegar a \textbf{Lady Gaga} y finalmente usamos su ``voice type'' para obtener \textit{mezzo-soprano}.

\begin{figure}[h!]
	\centering
	\includegraphics[width = 1\textwidth]{Imagenes/Bitmap/Voz ejemplo.png}
	\caption{Ejemplo de explicación por tipo de voz}
	\label{fig:sampleImage}
\end{figure}

\subsubsection*{Premios (Artista)}
Por último nos queda comentar la propiedad de premios. En esta sección hemos recogido dos propiedades: ``award received'' (premio recibido) y ``nominated for'' (nominado a). Estas categorías hacen referencia a los premios que ha recibido los artistas a lo largo de su carrera y la segunda a los premios a los que han sido nominados. Normalmente los premios se reparten según distintas categorías en función al estilo de música, pero también hay otros que nos aportan otro tipo de información al estudio como Grammy al Mejor Videoclip, Grammy Award a la Mejor Canción Escrita, Grammy Award a la Mejor Interpretación Rap/Sung etc...\\\\\\

\begin{figure}[h!]
	\centering
	\includegraphics[width = 1\textwidth]{Imagenes/Bitmap/awads.png}
	\caption{Ejemplo Relaciones que pueden tener artistas por premios}
	\label{fig:sampleImage}
\end{figure}


\subsection{Explicaciones basadas en el género}
El segundo estudio que hemos hecho ha sido el que viene ligado al género de la canción. Es una de las propiedades más detalladas en Wikidata y por lo tanto una de de las que más información nos pueden proporcionar. Hay muchos géneros y subgéneros documentados, lo que nos facilita la obtención y clasificación de estos.

En cuanto a esta propiedad, hemos sacado las siguientes relaciones:

\subsubsection*{Influenciado por y Subgénero}
Al igual que pasaba con los artistas, los géneros también son influenciados por otros los cuales ha servido para el desarrollo de nuevos estilos a lo largo de la historia. Gracias a la evolución del Rock hay otros géneros que han marcado tendencia como el Rock and Roll, el Heavy Metal,Hard Rock, Garage Rock etc...
Cabe destacar la diferencia entre las propiedades Influed by y Subgenre, mientras que la primera solo indica que ese estilo musical ha marcado una cierta influencia tomando algunos aspectos, la segunda señala que es un Subgénero, una tendencia muy parecida y que toma como base ese estilo musical.

\subsubsection*{Fecha de fundación}

Con esta propiedad intentamos tener en cuenta la historia de la música o su cronograma. Hay géneros que fueron desarrollados conjuntamente a lo largo del tiempo. Un ejemplo de este caso se dio en curiosamente el tecnopop que tanto caracterizó a los años 80, pues convivió a veces con el soul y el estilo afroamericano doo-wop, de Billy Joel.

\subsubsection*{País de origen (Género)}

Por último, recogemos la información del país donde se formó ese género. Con esta propiedad queremos poder recoger movimientos artísticos que surgieron en el mismo país. Los géneros que fueron formados en un determinado país están mucho más arraigados a este a pesar de que se hayan extendido por todo el mundo. Es por esto que si queremos relacionar una artista que cualquier parte del mundo, que tiene un estilo Techno, será más propenso a estar relacionado con artistas o canciones Holandesas, cuna del este estilo musical.