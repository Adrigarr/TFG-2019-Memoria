\chapter*{Resumen}

\section*{\tituloPortadaVal}

Los sistemas de recomendación proporcionan a los usuarios opciones seleccionadas en base a sus preferencias, aportándoles una forma cómoda y rápida de encontrar elementos de interés. Para que este proceso sea efectivo, sin embargo, el usuario debe comprender las recomendaciones y confiar en el sistema para encontrar lo que busca. Esta confianza puede lograrse mediante el uso de explicaciones, que son una forma intuitiva de justificar la elección de las recomendaciones de forma comprensible para el usuario.\\

Este trabajo busca enriquecer las explicaciones para un recomendador de música gracias a los datos enlazados. Se hará un estudio de esta tecnología y las formas en que puede aplicarse al ámbito de los sistemas de recomendación y se demostrará de forma práctica con el desarrollo del prototipo de una aplicación que obtenga y muestre explicaciones para el recomendador mencionado.\\


\section*{Palabras clave}
   
\noindent Datos enlazados, web semántica, explicaciones, RDF, SPARQL.

   


