\chapter{Reparto de trabajo}
\label{Appendix:Key1}

\section{Adrián Garrido Sierra}

En lo que respecta a la realización de esta memoria, Adrián se ha encargado de realizar las siguientes tareas:

\begin{itemize}
\item Formato y cohesión mediante el uso de LaTeX.

\item Referencias bibliográficas.

\item Revisión y colaboración en el Capítulo 1: Introducción.

\item Revisión y colaboración en el Capítulo 2: Estado de la Cuestión.

\item Capítulo 3: Tecnologías y Herramientas utilizadas.

\item Aportación principal del Capítulo 4: Explicaciones.

\item Capítulo 5: Diseño de la interfaz de explicaciones.

\item Colaboración en el apartado 6.3 Arquitectura de la aplicación.

\item Sección 7.1 Conclusiones\\
\end{itemize}

Pasando a la implementación, ha sido el responsable principal de las siguientes partes del trabajo:

\begin{itemize}
\item Montaje del framework y el servidor. Esto incluye el alojamiento en Heroku.

\item Organización de la estructura de carpetas.

\item Programación de las vistas, incluyendo HTML, CSS y JavaScript relacionado.

\item Programación del controlador.

\item Programación de funciones auxiliares para el algoritmo encargado de generar los grafos de explicaciones a partir de los datos obtenidos del modelo.





\section{Diego Sánchez Muniesa}

Los siguientes puntos de la memoria han sdo elaborados por parte de Diego Sanchez Muniesa:

\begin{itemize}
\item Formato y cohesión mediante el uso de LaTeX.

\item Redacción del Capítulo 1 del proyecto: Introducción.

\item Redacción del Capítulo 2: Estado de la Cuestión.

\item Revisión y aportaciones del Capítulo 4.

\item Aportación principal del Capítulo 4: Explicaciones.

\item Capítulo 6: Redacción del Capítulo 6: Arquitectura de la aplicación .\\

\end{itemize}

En implementación, ha sido el responsable principal de las siguientes partes del trabajo:

\begin{itemize}

\item Creación del dataset limpio de canciones que se toma como source en la aplicación mediante su preprocesamiento, parseo y adaptación a los estándares de Wikidata.

\item Estructura del sistema de unión y relación de los datos mediante un modelo linked data.

\item Codificación y montaje de queries mediante la librería SPARQLWrapper.

\item Estructura y codificación del modelo de la aplicación.

\item Programación de scripts auxiliares a modo de pruebas.

\end{itemize}
\end{itemize}
