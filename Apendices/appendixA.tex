\chapter{Reparto de trabajo}
\label{Appendix:Key1}

La toma de decisiones en las distintas áreas del proyecto, así como la distribución de tareas, ha sido un esfuerzo consensuado entre los miembros del equipo. Para asegurar esta cooperación, ambos miembros han hecho una serie de reuniones informales (generalmente cada una o dos semanas) para poner en común el progreso individual realizado y acordar el curso a seguir a partir de ahí.

\section{Adrián Garrido Sierra}

Durante las fases iniciales del proyecto, Adrián participó en la investigación relacionada con la Web Semántica y Linked Data, así como en la búsqueda y selección de la base de conocimiento a utilizar (que finalmente sería Wikidata) y la librería para realizar consultas en la aplicación a desarrollar. \\

Contribuyó activamente a la elaboración de la lista de explicaciones a estudiar, tiempo en que también se familiarizó con el uso de Wikidata y su sistema de consultas SPARQL. \\

En lo que respecta al desarrollo del código de la aplicación, fue el responsable de la selección y montaje del framework y el servidor, incluyendo el alojamiento en la plataforma Heroku que es la que se utiliza para presentar la aplicación a los usuarios. \\

Fue el encargado principal de la organización de la estructura del código siguiendo el patrón MVC~(Modelo-Vista-Controlador) y también de la programación íntegra de los componentes de la Vista y del Controlador según el mismo patrón. \\

Siguiendo con lo mencionado, Adrián ha sido el principal encargado de diseñar y desarrollar la interfaz de explicaciones utilizada en la aplicación. Este ha sido un proceso iterativo en el que hubo que considerar diversas opciones tanto a nivel de diseño como de implementación. \\

Centrándose en la representación de los grafos, investigó varias librerías disponibles relacionadas con la elaboración de gráficos y, tras probar y desechar alguna (como Alchemy.js), finalmente implementó al trabajo vis.js. Junto con ello también programó todo lo necesario para el tratamiento de los datos de las explicaciones con el fin de generar las estructuras necesarias para dibujar los grafos con esta librería, además de todas las funciones auxiliares para su correcta visualización. \\

Pasando a la elaboración de este documento de Memoria, Adrián ha hecho numerosas aportaciones. Utilizó \LaTeX y concretamente la plantilla TeXiS para dar formato y coherencia al documento, además de encargarse de la gestión de las referencias bibliográficas gracias al uso del programa JabRef. \\

Escribió íntegramente el Resumen y el Capítulo~\ref{cap:interfaz}. Diseño de la interfaz de explicaciones debido al desarrollo realizado en la interfaz, además de hacer todas las traducciones al inglés encontradas en el Capítulo~\ref{cap:introduction}. Introduction y \ref{cap:conclusions}. Conclusions and Future Work. \\


En el Capítulo~\ref{cap:implementacion}. Implementación se encargó de buena parte de la sección de Arquitectura de la aplicación y  también añadió una lista de herramientas y tecnologías estudiadas durante el desarrollo de la aplicación al final de este mismo capítulo. \\

También es el autor de la sección de Conclusiones en el Capítulo~\ref{cap:conclusiones}. Conclusiones y Trabajo Futuro y de una buena parte del Capítulo~\ref{cap:explicaciones}. Explicaciones, incluyendo la selección y explicación de los distintos ejemplos presentados. \\

Por último, hizo grandes contribuciones a los capítulos \ref{cap:introduccion}. Introducción y \ref{cap:estadoDeLaCuestion}. Estado de la Cuestión en forma de varias revisiones y ampliaciones para llevarlos a un estado satisfactorio tanto en formato como en contenido. \\


\section{Diego Sánchez Muniesa}

Inicialmente, comencé estudiando todo lo que hace referencia al estudio del arte y sus principales conceptos: ¿Qué es la Web Semántica?, ¿qué es el concepto de Linked Data y como se relacionan?, ¿cómo se estructuran los datos y cómo acceder a ellos? y demás información referente al estado del arte.\\

Posteriormente, estuve investigando posibles opciones a elegir para poder establecer la base de datos que íbamos a usar, junto a su correspondiente plataforma o API final. Era obviamente necesario que mantuviese un modelo RDF para poder crear relaciones o links entre toda la información que necesitábamos relacionar. Entre ellas he podido testear principalmente DBpedia y MusicBrainz hasta que finalmente optamos por lo que consideramos la mejor opción: Wikidata.\\

Una vez seleccionada la base de datos que íbamos a usar en el proyecto, comencé a crear un conjunto de queries que nos brindasen información sobre canciones aleatorias, con el propósito de obtener soltura y experiencia con el lenguaje SparQL y su aplicación en Wikidata ya que tiene un manejo diferente a otras bases de datos.\\

Más tarde, me encargué de pensar un sistema que pudiese unir diferentes explicaciones de forma que, al aplicar un proceso de obtención de datos, estos pudiesen almacenarse ordenada y jerárquicamente en nodos y aristas. Esta necesidad era un punto clave ya que de no tener una estructura adecuada, la representación gráfica sería perjudicada. Con queries individuales no podría hacerse un trabajo complejo y eficiente. El primer punto fue la modificación de las prestaciones de la librería de SparqlWrapper en un módulo el cuál nos permitía la conexión con la API de Wikidata y el manejo de queries y sus respectivas respuestas. Alrededor de ese módulo se creó el modelo de la aplicación, haciendo un estudio de todas las explicaciones que nos habíamos fijado en un proceso en el cuál, se reciben dos canciones de entrada y en consecuencia, se inicia un proceso encargado de hacer todo el estudio secuencialmente hasta crear un esquema de todo lo compartido por ambas canciones.\\

Después elaboré un pre-procesado de la lista inicial de canciones de LastFM el cual contaba con una larga lista de alrededor de 19 millones de canciones, muchas de ellas, repetidas o con valores incorrectos y nulos. Quedó un final de 421 canciones, después de un tratado de sus caracteres y de hacer el testeo de que podíamos obtener información relevante sobre ellas en Wikidata.\\

Por último, me encargué de elaborar pruebas del modelo y sus consiguientes actualizaciones a medida que obteníamos distintos datos en diferentes casos de uso, ya que, no siempre el formato de los datos recogidos era el correcto o a veces surgían excepciones por la falta de ellos en la base de datos.\\
